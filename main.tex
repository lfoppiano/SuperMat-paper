\documentclass[a4paper,10pt]{article}
\usepackage[utf8]{inputenc}
\usepackage{authblk}
\usepackage{tabularx}
\usepackage{url}

\title{SuperMat: Construction of a linked annotated dataset from superconductors-related literature}
% \title{SuperMat: Construction of a linked annotated dataset for superconductors material discovery}
% \title{SuperMat: Construction of a linked annotated dataset to support superconductors material research}

\author[1]{Luca Foppiano\thanks{FOPPIANO.Luca@nims.go.jp}}
\author[1]{Sae Dieb}
\author[1]{Akira Suzuki}
\author[2]{Pedro Baptista de Castro}
\author[2]{Suguru Iwasaki}
\author[2]{Yan Meng}
\author[2]{Terashima Kensei}
\author[2]{Yoshihiko Takano}
\author[1]{Masashi Ishii\thanks{ISHII.Masashi@nims.go.jp}}

\affil[1]{Material Database Group, MaDIS, NIMS, Japan}
\affil[2]{Nano Frontier Superconducting Materials Group, MANA, NIMS}

\begin{document}

\maketitle

\begin{abstract}
%The creation of large collaborative projects such as the Materials Genome Initiative~\cite{material_genome_initiative}, contributed to accelerate the change, however the amount of structured data represent a negligible part of the potentially available information.

%At the moment, the only available database for superconducting material discovery is Supercon, which is old, small (only 30000 entries) and outdated. There is needs of modernising or rebuilding the current manual process, which cannot keep-up with the current publication rate. 
% We are currently working on a collaborative project to implement a system for mining information from superconductors-related literature using different approaches of machine learning and rule-based. 

% We annotated entities such as materials, classes, measurement methods, and properties such as superconducting critical temperature values and expressions, and critical pressure. We linked materials with critical temperature, pressures, and measurement methods. 


% importance of text and data mining in material research (max 2 sentences) 
In materials science, abundant publications are available as knowledge source. However, information provided as text in a non-standardised and unstructured form, is difficult to be used as machines readable media.
The establishment of Text and data mining (TDM) processes is the missing building block to obtain big structured data, needed to accelerate the research through data-driven exploration~\cite{Hill2016MaterialsSW}.
% Importance of superconducting material research 
Although superconductors discussed in this work are already used as components of quantum computer, medical instruments, electric power transmission, among others, the research to discover higher temperature superconductors is still active. 
[TODO add example of superconductors research]
TDM for extracting structured data indicating relations between samples, properties, and conditions is playing an important role in data-driven research to support breakthrough in the high temperature superconductor discovery.
% other attempts 
However, only few attempts of TDM from materials science literature have been described by \cite{Dieb2011ConstructionOT}, \cite{court2018auto} and \cite{kononova_text-mined_2019}. 
We propose SuperMat (Superconductor Materials), a linked dataset of annotated data from scientific literature, as a result of a collaborative project between computers and materials scientists.  
Our work counts several tasks: (1) relevant information selection, (2) tag-set and relationship design, (3) annotation rules defintion, and (4) validation by domain experts.
The dataset, composed by 114 articles and 9966 entities, contains annotations of materials, samples, classes, properties, and conditions. 
We provide links of superconducting critical temperature (\textit{Tc}) with parametric conditions such as pressure values, and the used methods of measurements, when specified.
% Usage
This dataset can be applied to train supervised models or for end to end evaluation, regardless the approach. 
On a broader scope, it can be utilised for document summarisation, clustering or information retrieval.  [TODO: Rephrase]
    
\end{abstract}

\newpage

\section{Background and summary}
% Introduction, why text is important for scientific knowledge? 
The vast majority of scientific knowledge is available as text,  publication rate increasing every year at an overwhelming pace~\cite{Grigas2017JustGI} \cite{Khabsa2014TheNO} \cite{OrduaMalea2015MethodsFE} \cite{Bjrk2009ScientificJP}.  In some highly productive fields, e.g. nanotechnologies, the number of published papers doubles every year~\cite{Youtie2008NanotechnologyPA}. 

Publications, however, are presented in an non-standardised and unstructured form, difficult to be used as machine readable media. 

Numerous initiatives focusing on automatic document processing, through Text and Data Mining (TDM) processed  have been established,  document processing for obtain specific  information such as document synthesis, entities extraction or clustering. 

[TODO: rewrite]


about materials, experiments and properties, 

In materials science, the  data-driven approach applied to material discovery has been facing several challenges~\cite{Hill2016MaterialsSW}. 
Nevertheless, it has become increasingly popular in the last decades thanks to large collaborative projects such as the Genome Initiative~\cite{material_genome_initiative} \cite{Jain2013CommentaryTM_materialsProject}, and the general availability of tools and computing power. 
% Dieb: Why collecting information from text is useful for material science? 
However, the availability of large domain-specific databases has become a required step to create solid bases for data-driven exploration. 
Beside all the effort, the amount of structured and organised data still represents a negligible part of the total amount of the available published knowledge.
Most of the largely used material databases, such as Supercon~\cite{SuperCon}, Polynfo~\cite{polynfo}, the Pauling File~\cite{Blokhin2018ThePF_paulingFile}, have been manually constructed and curated over decades. Manual curation, in many sub-domain, is the safest approach to ensure the creation of high-quality data and this process is becoming both technically and economically unsustainable. 
[TODO: rewrite]

% Superconductors domain
%   - concrete example of superconducting case -> which properties or information can be useful for superconducting material scientists? [Suzuki/Terashima/Pedro]

High-temperature superconductors have many promising applications, however, to discover a new superconductor is very difficult, it is said that only 3\% of candidate materials is actually a superconductor~\cite{Konno2018DeepLO}.
Supercon has been created to support this research using data-driven approach, although the effort to manually create the database, the 30000 entries are a small number as compared with the totally published information. 
An automatic system is therefore necessary to be developed. 

%[TODO: introduce the use of Matchine Learning]
We decided to utilise an approach based on Machine Learning, which provides higher tolerance to noise and faster generalisation capabilities than classical rule-based approaches. This choice comes at the price of having data for training the probabilistic model. Looking at the scientific literature, this specific needs is an unexplored terrain. There are no records of previously made datasets in the superconductors domain. 
%   - needs of information in superconductors
%   - the available databases are a) old, b) small, c) manually constructed -> outdated
%   - automatic extraction is required 
% Dieb: Why the construction of the corpus is useful? 
%   - what is the related work? mention other corpora 
%   - need of training data for creating the model / system 
In such unexplored terrain, there is no record of previous attempts in the scientific literature, nor existing datasets in the public domain. 
In this submission, we present, in collaboration with the Nano Frontier Superconducting Material Group, our work and methodology for constructing a corpus for superconductor material data: SuperMat (Superconductors Materials). At the time of writing this paper, we have annotated 114 articles, with 9966 entities divided into six classes: material, class, critical temperature expressions, critical temperature value, critical pressure value and measurement method. Described in section \ref{sec:method} more in details. 

The paper is structured as follows: first we discuss the methodology used to obtain the data and produce the final result. Then we describe the information and the validation of the data contained in the dataset. Finally, we provide example of applications, usability and availability information. 

% [END]
% -- 

% [OLD] Data-driven science has become as the fourth dimension in scientific exploration, after experimentation, theory and simulation~\cite{doi:10.1063/1.4944682}.


% [OLD] The emergence of Machine Learning (ML), a sub-field of Artificial Intelligence (AI), followed by a growing infrastructure of tools for generating, testing, and refining scientific models give us some hope. Such approach allow to address complex problems which conventional techniques cannot solve efficiently. 

% Introduce the importance of high-quality training data 
% [OLD] On the other hand the development of statistical models require a solid base where to build more complex systems. Low quality or incorrect data, will propagate and exponentially impact on the out result. One of the dogma of text processing is "Garbage in, Garbage out", therefore is foremost important to reduce at minimum the "Garbage in" by having high quality data. 


% to be rephrased
%Contrary to what might seem like the conventional machine learning mantra, throwing more data at the problem is not always the solution. Instead, the quality and domain-specificity of the corpus determine its effectiveness for domain-specific tasks.

% In the field of superconductors materials, the manual data collection used to populate SuperCon\footnote{\url{http://supercon.nims.go.jp}} cannot cope with the massive fresh information from the increasing number of articles published every year. In addition, the current process cannot easily scale infinitely, for physical and economical constraints. 
%- 

%-

% A project is currently ongoing, and aims to develop a system for automatic superconductors database creation. From large quantities of superconductors{-}related articles, it aims to extract, automatically, superconductors material and their relative properties~\footcite{foppiano2019proposal}. 

% The dataset provides scientific text annotated with entities and relationship information (links). The entities are identified among 6 classes (or labels) as summarised in Figure~\ref{fig:classes_frequency} and linked using relationships \texttt{material-tc} and \texttt{tc-pressure} (Table~~\ref{tbl:summary-links}). 

% This corpus is designed for training sequence labelling statistical models and can be utilised for developing domain-specific systems for entity extraction, entity-relationship and clustering. 

\label{sec:method}
\section{Method}
\subsection{Content acquisition}
Our content is composed by scientific articles. We collected the content from several sources. (a) the Open Access version of articles referenced in the SuperCon database records. (b) Articles of interests according with domain-experts research topics. (c) Articles from the arXiv's “Condensed matter” category\footnote{\url{https://arxiv.org/archive/cond-mat}} obtained, by searching with keywords such as 'superconductor', 'critical temperature' and 'superconductivity'.

The corpus was collected in batches, which are simpler to manage, with the growing dimension. 

\subsection{Tag-set design}
The tag-set represent the labels and relative information that we want to capture. We discussed with domain-experts, the end clients of our system, what are the relevant information and entities they are interested in, in order of importance. We wanted to find the optimal trade-off that could keep the system simpler, yet providing all main needed information for domain experts. 

As a result, we compiled a summary table (Table \ref{table:summary-entities-superconductor}) of relevant entities, in order of importance for domain experts [TODO: waiting for an prioritised version of this table from Takano Gr. ].

\begin{table}[h!]
    \centering
    \begin{tabular}{ | m{5em} | m{8cm}| m{5em} | } 
    \hline
        Name & Description & type \\ [0.5ex] 
    \hline\hline
        T\textsubscript{c} & Superconducting critical Temperature & result\\ 
    \hline
        T\textsubscript{onset} & Temperature at which the electrical resistance starts to decrease sharply & result\\
    \hline 
        T\textsubscript{offset} & Temperature at which zero resistance starts to appear & result\\ 
    \hline
        P & Superconducting critical pressure & condition\\
    \hline
        Material & Material composing the sample & sample \\
    \hline  
        Substrate & Material layers composing the sample & sample\\
    \hline
        Crystal structure space group & Crystal configuration (single crystal, poly-crystal) & sample \\
    \hline 
        Preparation shape & wire, powder, crystal, thin film & sample\\
    \hline 
        Measurement method & Method used for measuring the superconducting critical temperature (theoretical/calculation, electrical resistivity, magnetic susceptibility and specific heat) & measurement\\
    \hline
        H\textsubscript{c1} & Lower Critical field & condition\\ 
    \hline
        H\textsubscript{c2} & Higher Critical field & condition\\ 
    \hline
        I\textsubscript{c} & Critical current & condition\\
    \hline
        J\textsubscript{c} & Critical current density & condition\\ 
    \hline
        H\textsubscript{ivr} & Irreversibly field & condition\\
    \hline    
        Fabrication process & process used to fabricate the sample & sample\\
    \hline    
    \end{tabular}
    \caption{Summary of the relevant entities in superconductor research papers}
    \label{table:summary-entities-superconductor}
\end{table}

We decided to focus on extracting information from text as this first initial stage. Table or plot extraction can be run as separate parallel projects.  

\subsection{Annotation process}
\label{sec:annotation-process}
We describe the annotation process as a variation of the MATTER approach: Model, Annotate, Train, Test, Evaluate, and Revise, which was initially described by [ADD CITATION]. 

Our system uses already established open-source libraries which are providing already several functionalities which can simplifying our workflow. Our workflow consists in the following steps: Pre-annotate, Correct annotation, Validate, Train/Test/Evaluate, Revise. 

We are under the assumption of having a system implementing some sort of recognition (a machine learning model trained with one document or a rule-based approach) to bootstrap the process. 

The data is pre-annotated, then annotators can correct these annotations. Once a document is completed, it is validated by domain experts, which are checking the document again. This process brings two benefits: reduce distraction errors, and validate the annotation from the domain point of view. 
[TODO: add schema summarising this process]

The pre-annotation 

The correction


\section{Data Record}
Description of the analysis of the dataset 

\section{Applications}
How the corpus can be used

\section{Technical Validation} 
Write about the results of the comparison between 
- CS vs DE 
- MS vs DE
- Kubokawa-san / Meng-san vs DE? 

\section{Data Usage} 
Describe how the data is provided XML, JSON... 

\section{Code Availability}
where, how the data will be distributed 

Github? MDR, Nomad? 


\section{Conclusion}
Conclusions and future work 

% IAA
% We have analysed the process and the evolution of the agreement during our process and how this can be used to understand when certain constraints or definition are to be clarified further. 
On the creation of annotation we should improve the process of creating training data, especially when the domain expert will be actively involved in task. 

\bibliographystyle{unsrt}
\bibliography{references}  

\end{document}
